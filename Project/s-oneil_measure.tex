\subsection{Measures with Stranger Tangent Measures}

In the previous chapter on rectifiability, we saw examples of measures whose tangent measures are well-behaved. In particular, all tangent measures $\tau$ we saw satisfied (at least almost everywhere)
\begin{equation}
    \tau(B(p,r)) = r^\alpha \tau(B(p,1))
\end{equation}
for some $\alpha > 0$. Such measures are called $\alpha$-\textit{uniform}. In this chapter, we will demonstrate that this property is not uniform (haha) among all tangent measures. First, for any arbitrary measure $\tau$ and a point $p \in \bbR^d$, we will define a measure $\mu$ which has $\tau$ as a tangent measure at $p$.

Fix $\tau \in \scrM(\bbR^d)$ and $p \in \bbR^d$. We will show there exists $\mu \in \scrM(\bbR^d)$ with $\tau \in \Tan(\mu,p)$. Fix $\epsilon \in (0,1)$, set $s_1 := 1$, and inductively define $s_j := \epsilon^j s_{j-1}$. Define $\mu \in \scrM(\bbR^d)$ by
\begin{equation}
    \int_{\bbR^d} \phi \; \d\mu := \int_{\bbR^d} \phi\parens{ \sum_{i=1}^\infty x s_i } \; \d\tau(x) \quad \text{for all } \phi \in C_c(\bbR^d).
\end{equation}
We then have 
\begin{equation}
    \int_{\bbR^d} \phi \; \d\mu_{p,s_j} = \int_{\bbR^d} \phi\parens{ \sum_{i=1}^\infty (x - p)\frac{s_i}{s_j} } \; \d\tau.
\end{equation}
The terms in the sum with $j > i$ become arbitrary large (on the order of at least $\epsilon^{-j}$) as $j \to \infty$. Similarly, the terms in the sum with $j < i$ become arbitrarily small (on the order of at most $\epsilon^{j+1}$ as $j \to \infty$. By continuity and compact support of $\phi$, we may take $j \to \infty$ to see that this converges to
\begin{equation}
    \int_{\bbR^d} \phi \; \d\tau.
\end{equation}
Therefore $\tau$ is a tangent measure to $\mu$ at $p$. 

\subsection{The O'Neil Measure}
All measures we have constructed so far have had unique tangent measures up to multiplication. Do there exist measures $\mu$ such that $\Tan(\mu,p)$ contains distinct tangent measures for some point $p \in \bbR^d$? There certainly are - example 14.2(3) in \textbf{[Mattila]} is one. In this section, we will construct a measure $\mu$ such that $\Tan(\mu,p)$ at $\mu$-a.e. point $p \in \bbR^d$ is all of $\scrM(\bbR^d)$. Naturally, this is the worst it can possibly be, although we will see later on that this is not actually uncommon for Radon measures. 

The following theorem was introduced by Toby O'Neil in \textbf{[O'Neil]}, and the proof will follow his method.
\begin{theorem}[O'Neil] \label{thm:oneil}
    Fix $d \in \bbN$. There exists a Radon measure $\mu \in \scrM(\bbR^d)$ such that for $\mu$-a.e. $p \in \bbR^d$, we have $\Tan(\mu,p)$.
\end{theorem}
\begin{proof}
    Define the set $\scrS \subseteq \scrM(\bbR^d)$ to be the set of convex linear combinations of Dirac measures $\delta_x$ for $x \in \bbQ^d \cap B(0,1)$, with $\delta_0$ always included in the convex linear combination. That is,
    \begin{equation} \label{eq:setOfMeasuresForOneilMeasure}
        \scrS := \left\{ \alpha_0 \delta_0 + \sum_{i=1}^N \alpha_i \delta_{x_i} \;\middle|\;
        \begin{aligned} 
            &N \in \bbN, \alpha_i \in \bbQ \cap (0,1), \sum_{i=0}^N \alpha_i = 1, \\
            &x_i \in \bbQ^d \cap B(0,1), x_i \neq x_j \text{ for } i \neq j 
        \end{aligned}
        \right\}.
    \end{equation}
    We show that the set $\{ p\nu_{0,q} : p,q \in \bbQ^+, \nu \in \scrS \}$ is weakly* dense in $\scrM(\bbR^d)$. First, note that the set of measures with compact support in $\scrM(\bbR^d)$ is dense in $\scrM(\bbR^d)$. This can be shown, for example, by considering the compactly supported measures $\mu \restrict B(0,N)$ for $N \in \bbN$. 
    Fix, therefore, a compactly supported $\mu \in \scrM(\bbR^d)$. Given $m \in \bbN$, let $\scrQ_m$ be the family of half-open cubes in $\bbR^d$ with side length $\frac{1}{m}$ congruent to $Q_0 = [-\frac{1}{2m},\frac{1}{2m})$. Let $x_Q$ be the midpoint of $Q \in \scrQ_m$. Certainly, we have that
    \begin{equation} \label{eq:riemsum}
        \begin{aligned}
            \abs{ \int_{\bbR^d} f \, \d{\mu} - \sum_{Q \in \scrQ_m} f(x_Q)\mu(Q) } 
            &= \abs{ \int_{\bbR^d} f - \sum_{Q \in \scrQ_m} f(x_Q) \mathbbm{1}_Q \, \d{\mu} } \\
            &\leq \int_{\bbR^d} \sum_{Q \in \scrQ_m} \abs{ f(x) - f(x_Q) } \mathbbm{1}_Q(x) \, \d{\mu(x)} \\
            &\leq \int_{\bbR^d} \sum_{Q \in \scrQ_m} \abs{x-x_Q} \mathbbm{1}_Q(x) \, \d{\mu(x)} \\
            &\rightarrow 0 \text{ as } m \to \infty
        \end{aligned}
    \end{equation}
    for all Lipschitz $f$ with compact support and with Lipschitz constant at most 1. The latter integral converges to 0 by the dominated convergence theorem, since $\mu$ has compact support.
    Fix $\epsilon > 0$, and choose $m_0 \in \bbN$ such that $m \geq m_0$ implies 
    \begin{equation} \label{m0definition}
        \abs{ \int_{\bbR^d} f \, \d{\mu} - \sum_{Q \in \scrQ_m} f(x_Q)\mu(Q) } < \epsilon
    \end{equation}
    for all $f \in \Lip_{\leq 1}(\bbR^d)$. Such a choice is possible by \eqref{eq:riemsum}.
    Since $\mu$ has compact support, there exists $N \in \bbN$ such that $\mu(Q) = 0$ whenever $Q \cap B(0,N) = \emptyset$, so that 
    \begin{equation} 
        \sum_{Q \in \scrQ_m} f(x_Q)\mu(Q) = \sum_{\substack{Q \in \scrQ_m \\ Q \cap B(0,N) \neq \emptyset}} f(x_Q)\mu(Q)
    \end{equation}
    Write $\widetilde{\scrQ}_m = \{ Q \in \scrQ_m : Q \cap B(0,N) \neq \emptyset \}$.
    Next, index the elements $Q_i$ of $\scrQ_m$ by $i \in \bbN \cup \{0\}$, with $Q_0$ being the cube centered at 0 as above. For each $i \in \bbN \cup \{0\}$, let $p_{Q_i} \in \bbQ^+$ be such that $\abs{ \mu(Q_i) - p_{Q_i} } < \frac{\epsilon}{2^{i+1}}$. In particular, if $Q_i \notin \widetilde{\scrQ}_m$, we can set $p_{Q_i} = 0$. We must, however, have that $p_{Q_0} > 0$. Similarly, for each $i \in \bbN \cup \{0\}$, let $x_{Q_i}^* \in \bbQ^n$ be such that $\abs{ x_{Q_i} - x_{Q_i}^* } < \frac{\epsilon}{2^{(i+1)} p_{Q_i} }$. We then have
    \begin{equation}
        \begin{aligned}
            \abs{ \sum_{Q \in \scrQ_m} f(x_Q)\mu(Q) - \sum_{Q \in \scrQ_m} f(x_Q^*)p_Q }
            &\leq \abs{ \sum_{Q \in \scrQ_m} f(x_Q)\mu(Q) - \sum_{Q \in \scrQ_m} f(x_Q)p_Q }\\
            &\hspace{20pt} + \abs{ \sum_{Q \in \scrQ_m} f(x_Q)p_Q - \sum_{Q \in \scrQ_m} f(x_Q^*)p_Q } \\
            &\leq \sum_{Q \in \scrQ_m} f(x_Q) \abs{ \mu(Q) - p_Q }
                + \sum_{Q \in \scrQ_m} \abs{ f(x_Q) - f(x_Q^*) }p_Q \\
            &\leq \norm{f}_\infty \sum_{i=0}^\infty \frac{\epsilon}{2^{i+1}}
                + \sum_{Q \in \scrQ_m} \abs{ x_Q - x_Q^* }p_Q \\
            &\leq \norm{f}_\infty \epsilon + \sum_{i=0}^\infty \frac{\epsilon}{2^{i+1} p_{Q_i}} p_{Q_i} \\
            &= \epsilon(1 + \norm{f}_\infty).
            \end{aligned}
    \end{equation}
    for all Lipschitz $f$ with compact support and with Lipschitz constant at most 1. 
    
    For each $Q \in \widetilde{\scrQ}_m$, set $y_Q^* := \frac{x_Q^*}{N+1} \in \bbQ^n \cap B(0,1)$. Then 
    \begin{equation}
        \begin{aligned}
            \sum_{Q \in \widetilde{\scrQ}_m} p_Q \delta_{x_Q^*} 
            &= \sum_{Q \in \widetilde{\scrQ}_m} p_Q \delta_{Ny_Q^*} \\
            &= \left( \sum_{Q \in \widetilde{\scrQ}_m} p_Q \delta_{y_Q^*} \right)_{0,N^{-1}} \\
            &= \left( \sum_{Q \in \widetilde{\scrQ}_m} p_Q \right) \left( \sum_{Q \in \widetilde{\scrQ}_m} \frac{p_Q}{\sum_{Q \in \widetilde{\scrQ}_m} p_Q} \delta_{y_Q^*} \right)_{0,N^{-1}} \\
            &=: p \nu_{0,N^{-1}}.
        \end{aligned}
    \end{equation}
    Let $f \in \Lip_{\leq 1}(\bbR^d)$. By our calculations above, we have that for $m \geq m_0$ (where $m_0$ was defined in \eqref{m0definition}),
    \begin{equation}
        \begin{aligned}
          \abs{ \int_{\bbR^d} f \, \d{\mu} - \int_{\bbR^d} f \, \d{(p\nu_{0,N^{-1}})} }
          &= \abs{ \int_{\bbR^d} f \, \d{\mu} - \sum_{Q \in \widetilde{\scrQ}_m} f(x_Q^*) p_Q } \\
          &= \abs{ \int_{\bbR^d} f \, \d{\mu} - \sum_{Q \in \scrQ_m} f(x_Q^*) p_Q } \\
          &\leq \abs{ \int_{\bbR^d} f \, \d{\mu} - \sum_{Q \in \scrQ_m} f(x_Q)\mu(Q) } \\
          &\hspace{20pt} + \abs{ \sum_{Q \in \scrQ_m} f(x_Q)\mu(Q) - \sum_{Q \in \scrQ_m} f(x_Q^*)p_Q } \\
          &\leq \epsilon + \epsilon(1 + \norm{f}_\infty) \\
          &= \epsilon(2 + \norm{f}_\infty).
        \end{aligned}
    \end{equation}
    By construction, $\nu$ is an element of $\scrS$. Since $\epsilon > 0$ was arbitrary, we are done.

    Because of this and lemma \ref{lem:tangent_measure_properties}, it suffices to find a measure $\mu$ such that $\scrS \subseteq \Tan(\mu,p)$ for $\mu$-a.e. $p \in \bbR^d$. Indeed, lemma \ref{lem:tangent_measure_properties} parts (a) and (b) mean the set $\{ c\nu_{0,r} : c,r>0 \}$ is contained in $\Tan(\mu,p)$, and by the above analysis, this means $\Tan(\mu,p)$ is dense in $\scrM(\bbR^d)$. By part (c) of lemma \ref{lem:tangent_measure_properties}, we infer that $\Tan(\mu,p)$ is all of $\scrM(\bbR^d)$ for $\mu$-a.e. $p \in \bbR^d$. 

    Note that $\scrS$ is a countable set. Index the elements $\widetilde{\nu}_i$ of $\scrS$ by $i \in \bbN$.
    Define the sequence $(i_k)_{k \in \bbN \cup \{0\}}$ by 
    \begin{equation}
        \begin{aligned}
            i_0 &= 0; \\
            i_k &= k - \frac{1}{2}n(n+1) \text{ for } k \in \left[ \frac{1}{2}n(n+1), \frac{1}{2}(n+1)(n+2) \right), n \in \bbN.
        \end{aligned}
    \end{equation}
    That is, $i_k$ is the sequence $(1,1,2,1,2,3,1,2,3,4,\dots)$. If $k_n = \frac{1}{2}n(n+1)$ for $n \in \bbN \cup \{0\}$, then $i_{k_n} = 1$. More generally for $m \in \bbN$, define
    \begin{equation}
        k_m(n) = \frac{1}{2}(n+m-1)(n+m) + (m-1) \text{ for } n \geq 0.
    \end{equation}
    Then $k_m(n) = m$ for $n \geq 0$. We set $\nu_k := \widetilde{\nu}_{i_k}$. It follows that every element of $\scrS$ occurs infinitely many times in the sequence $(\nu_k)$. In particular, $\nu_{k_m(n)} = \widetilde{\nu}_m$ for all $n \geq m-1$.

    For each $k \in \bbN$, we write 
    \begin{equation}
        \nu_k = \alpha_k(0) \delta_0 + \sum_{j = 1}^{N_k} \alpha_k(x_{k,j}) \delta_{x_{k,j}}.
    \end{equation}
    For each $k \in \bbN$, define $\Omega_k := \{0,x_{k,1},\dots,x_{k,N_k}\}$. Each $\alpha_k$ can be interpreted as the probability mass function for the measure $\nu_k$ on $(\Omega_k, 2^{\Omega_k})$. We then set $\Omega := \prod_{k=1}^\infty \Omega_k$. Consider the $\sigma$-algebra $\scrA$ on $\Omega$ generated by the \textit{cylinder sets} of the form $\eta \vert_j = \{(\eta_1,\dots,\eta_j)\} \times \prod_{k=j+1}^\infty \{ x_{k,1},\dots,x_{k,N_k} \}$ for $\eta \in \Omega$, and define a probability measure $\alpha$ on $(\Omega, \scrA)$ by 
    \begin{equation}
        \alpha(\eta \vert_j) := \prod_{k=1}^j \alpha_k(\eta_k).
    \end{equation}
    That is, $(\Omega, \scrA, \alpha)$ is the product space $\prod_{k=1}^\infty (\Omega_k, 2^{\Omega_k}, \alpha_k)$.

    \begin{figure} 
        \begin{tikzpicture}
            %\draw[help lines] (0,0) grid (13,4);

            \foreach \j in {1,...,3}
            {
                \node at (2*\j - 1,.7) {$\Omega_\j$};
                \foreach \i in {1,...,30}
                {
                    \pgfmathsetmacro\radius{.8*rnd}
                    \draw[fill=red] (2*\j - 1,2) ++ (rnd*360:\radius) circle(0.05);
                }
            }
            \foreach \i in {1,2,3}
            {
                \draw[fill] (6,2) ++ (\i / 4,0) circle (0.03);
            }
            \draw[->,thick] (7,2) ++ (.1,.1) to [out=30, in=150] (8.9,2.1);
            \node at (8,2.7) {$f$};

            \coordinate (thebigone) at (11,2);
            \newsavebox{\cloud}
            \savebox{\cloud}
            {
                \foreach \i in {1,...,15}
                { 
                    \pgfmathsetmacro\radius{.6*rnd}
                    \draw[fill=green] (0,0) circle (0.05);
                    \draw[fill=purple] (rnd*360:\radius) circle(0.05*1/3);
                }
            }
            \foreach \i in {1,...,30}
            {
                \pgfmathsetmacro\radius{1.8*rnd}
                \draw[fill=purple] (thebigone) ++ (rnd*360:\radius) node {\usebox{\cloud}};
            }
 
        \end{tikzpicture}
        \caption{Mapping the product space $(\Omega,\scrA,\alpha)$ into $B(0,1)$.}
        \label{fig:oneilPushforwardMap}
    \end{figure}
     
    For each $k \in \bbN$, define 
    \begin{equation}
        \sigma_k := \min\set{ \abs{x - y} : x,y \in \Omega_k, x \neq y }.
    \end{equation}
    Inductively define the sequence $r_k$ by choosing $r_0 > 1$, and defining $r_k := \frac{r_0^{k}}{\sigma_k} r_{k-1}$.
    Define $f \colon \Omega \to B(0,1)$ by
    \begin{equation}
        f(\eta) := \sum_{k=1}^\infty \frac{\eta_k}{r_k}
    \end{equation}
    See figure \ref{fig:oneilPushforwardMap}. We set $\mu := f_* \alpha$ to be the pushforward measure of $\alpha$ by $f$ on the space $(\bbR^d, \scrB(\bbR^d))$. Our aim is to choose the $r_k$ so that as we zoom in to $\mu$-a.e. point in $B(0,1)$, we go through the sequence $\Omega_k$. In particular, we'd like the sequences $k_m$ to provide us with the different rates of zooming in.

    Given $m \in \bbN$, define
    \begin{equation}
        V_m := \{ \eta \in \Omega : \eta_{k_m(n)} = 0 \text{ i.o. in } n \}.
    \end{equation}
    That is, $V_m$ is the event of choosing the point $0$ infinitely often whenever $\nu_k = \widetilde{\nu}_m$.
    Now, since $\alpha(\eta_{k_m(n)} = 0) = \alpha_{k_m(n)}(0)$ is positive and constant with varying $n$, we have that 
    \begin{equation}
        \sum_{n=0}^\infty \alpha(\eta_{k_m(n)} = 0) = \infty.
    \end{equation}
    Since the events $\{ \eta_{k_m(n)} = 0 \}$ are independent, we conclude by the Borel-Cantelli lemma that $\alpha(V_m) = 1$. In other words, $\mu(f(V_m)) = 1$. Define $V = \bigcap_{m \in \bbN} V_m$ to be the event of selecting $0$ infinitely often along the whole sequence. By countability, $\alpha(V) = 1$.

    We will show that $\scrS \subseteq \Tan(\mu,p)$ for all $p \in f(V)$. Fix $p \in f(V)$, and write $p = f(\pi) = \sum_{k=1}^\infty \frac{\pi}{r_k}$ for some $\pi \in V$. Fix $m \in \bbN$, and let $n_j \in \bbN$ be a sequence with $n_j \rightarrow \infty$, and with $\pi_{k_m(n_j)} = 0$ for all $j$. This is possible by definition of $V$. Fix $\phi \in \Lip_{\leq 1}(\bbR^d)$. The proof will be concluded by lemma \ref{lem:equivalentDefinitionOfWeak*Convergence} once we show $\int_{\bbR^d} c_j \phi \; \d\mu_{\overline{x},s_j} \rightarrow \int_{\bbR^d} \phi \; \d\nu_m$ for some appropriate choice of $c_j$ and $s_j$. 
    
    Define the set $\Omega^{(j)} := \set{ \eta \in \Omega : \eta_k = \pi_k \text{ for } k = 1,\dots,k_m(n_j)-1 }$, and let $c_j = \alpha(\Omega^{(j)})^{-1}$. Also let $s_j = 1/r_{k_m(n_j)}$. We have
    \begin{equation}
        \int_{\bbR^d} \phi \; \d(c_j\mu_{\overline{x},s_j}) = c_j \int_{\bbR^d} \phi\parens{\frac{x - \overline{x}}{s_j}} \; \d\mu(x)
                                                            = c_j \int_{\Omega} \phi\parens{r_{k_m(n_j)} \sum_{k=1}^\infty \frac{\eta_k - \pi_k}{r_k}} \; \d\alpha(\eta).
    \end{equation}
    We also have 
    \begin{equation}
        \int_{\bbR^d} \phi \; \d\nu_m = \int_{\Omega_{k_m(n_j)}} \phi \; \d\nu_m = \int_\Omega \phi(\eta_{k_m(n_j)}) \; \d\alpha(\eta) 
        = c_j \int_{\Omega^{(j)}} \phi(\eta_{k_m(n_j)}) \; \d\alpha(\eta),
    \end{equation}
    where the last equality comes from the fact that the functions (random variables in this context) $\bbone_{\Omega^{(j)}}$ and $\eta \mapsto \phi(\eta_{k_m(n_j)})$ are independent.

    We would like to show, at least for $j$ large enough, that
    \begin{equation} \label{eq:restrictingAnIntegral}
        \int_{\Omega} \phi\parens{r_{k_m(n_j)} \sum_{k=1}^\infty \frac{\eta_k - \pi_k}{r_k}} \; \d\alpha(\eta) 
        = \int_{\Omega^{(j)}} \phi\parens{r_{k_m(n_j)} \sum_{k=1}^\infty \frac{\eta_k - \pi_k}{r_k}} \; \d\alpha(\eta),
    \end{equation}
    where $\Omega^{(j)} := \set{ \eta \in \Omega : \eta_k = \pi_k \text{ for } k = 1,\dots,k_m(n_j)-1 }$. Indeed, suppose this were true, and let $\eta$ be in this set. Then
    \begin{equation}
        \abs{ \eta_{k_m(n_j)} - r_{k_m(n_j)} \sum_{k=1}^\infty \frac{\eta_k - \pi_k}{r_k} } \leq  \sum_{k=k_m(n_j)+1}^\infty \frac{\abs{\eta_k - \pi_k}}{r_k}
        \leq \sum_{k=k_m(n_j)}^\infty \frac{1}{r_k},
    \end{equation}
    since all $\eta_k$ and $\pi_k$ are contained in $B(0,1)$, and $\pi_{k_m(n_j)} = 0$ by assumption. Since $\phi$ is Lipschitz with constant at most 1, this estimate implies
    \begin{equation} \begin{aligned}
        &\abs{ \int_{\bbR^d} \phi \; \d(c_j \mu_{\overline{x},s_j}) - \int_{\bbR^d} \phi \; \d\nu_m } \\
        &\quad = \abs{ c_j \int_{\Omega^{(j)}} \phi\parens{r_{k_m(n_j)} \sum_{k=1}^\infty \frac{\eta_k - \pi_k}{r_k}} \; \d\alpha(\eta)
                      - c_j \int_{\Omega^{(j)}} \phi(\eta_{k_m(n_j)}) \; \d\alpha(\eta) } \\
        &\quad \leq c_j \int_{\Omega^{(j)}} \sum_{k=k_m(n_j)} \frac{1}{r_k} \; \d\alpha \\
        &\quad = \sum_{k = k_m(n_j)}^\infty \frac{1}{r_k} \\
        &\quad \rightarrow 0,
    \end{aligned} \end{equation}
    noting that $\sum_{k=1}^\infty \frac{1}{r_k}$ certainly converges since $r_k$ increases very quickly.

    It remains to check (\ref{eq:restrictingAnIntegral}). Suppose $\eta \in \Omega \setminus \Omega^{(j)}$, and let $K < k_m(n_j)$ be the smallest integer with $\eta_K \neq \pi_K$. Let $R > 0$ be such that $\supp{\phi} \subseteq B(0,R)$. We have {\color{red} finish this. in particular, choose appropriate $r_k$}
\end{proof}

\subsection{Extensions and Conclusion}
Note that definition (\ref{eq:setOfMeasuresForOneilMeasure}) of the set $\scrS$ grants us some leeway in choosing $\mu$. Indeed, rather than taking $x_i \in \bbQ^d \cap B(0,1)$, we could ask that they come from a dense subset of another open ball, and the same proof would carry on over verbatim. One might wonder how restrictive the condition $\Tan(\mu,x_0) = \scrM(\bbR^d)$ actually is. All our examples in section 1 certainly didn't satisfy this, and it even took us a lot of work to show the existence of even one of these measures. As it turns out, this is not so restrictive at all. Tuomas Sahlsten in [Tangent Measures of Typical Measures,Tuomas Sahlsten,Real Analysis Exchange,Vol. 40, No. 1 (2014-2015), pp. 53-80] and Toby O'Neil in [A local version of the Projection Theorem and other results in Geometric Measure Theory,
PhD Thesis, University College London, 1994] independently managed to prove the following:
\begin{theorem} \label{thm:sahlstenTypicalMeasures}
    For a typical $\mu \in \scrM(\bbR^d)$, we have $\Tan(\mu,x_0) = \scrM(\bbR^d)$.
\end{theorem}
Here, \textit{typical} means this result holds for all $\mu$ in a residual subset (a countable union of dense open sets) of $\scrM(\bbR^d)$. The proof of theorem \ref{thm:sahlstenTypicalMeasures} is effectively a much more technical version of the construction above, making use of a set $\scrS$ similar to the above, and also explicitly defining the residual subset $\scrR \subseteq \scrM(\bbR^d)$.

{\color{red} hey billy can you use zorn's lemma to prove the o'neil theorem? i.e. say $\mu \prec \nu$ if there exists $N \subseteq \bbR^d$ with $\mu(N) = \nu(N) = 0$ and $\Tan(\mu,p) \subseteq \Tan(\nu,p)$ for all $p \in \bbR^d \setminus N$ then blah blah upper bound blah blah maximal element}